\newpage
\section{おわりに}
\label{sec:conclusion}

\subsection{まとめ}
本研究では3D-IWGANを用いて不整地を生成することで,走行計画に必要な学習用データを効率的かつ大規模に用意し,その走破性評価による実験フレームワークの作成を行った.3章で述べたとおり,生成した地形が走行計画の学習用データとしての有用性があるかは,統計的な特徴量だけで判断するのは難しく,実際にロボットを走行させて評価する必要があることが分かった.ここでは,マップ内をロボットに走行させた軌跡の含有率が近いデータが元の地形データと類似していると判断し,更にその有用性を評価できた.また,本研究での実験フレームワークによって元の月面データの走破結果に近い生成地形を用いることで,少数の地形データで走行計画に必要な機械学習用データを効率的に収集できることが分かった.


\subsection{今後の課題}
シミュレーション環境の実験フレームワークだけでなく,現実環境で実機を用いた検証を行えるような方法を検討する.また,本研究では月面を未知環境の例として取り扱ったが,災害現場等についても同様な実験が行えるか検討する.