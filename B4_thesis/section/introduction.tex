\section{はじめに}
\label{sec:introduction}
\subsection{研究の背景と目的}
現在,災害現場や月面などの未知環境におけるロボッ
トの活用が,人の活動領域の拡大やその際に起こりうる人の
身の危険性を回避するという点で期待されている. しかし,そ
れら未知の環境に対してロボットが自動操作のみで探査を行
うのは困難である. \cite{bunken5}また,それらの環境に対するシミュレー
ション用に地形データを収集するのにも限界がある. \cite{bunken0}そこで本
研究では,ロボットが走行する未知の不整地環境を効率的に
生成する方法と,その地形を評価するための走行実験フレー
ムワークを提案する.

% \subsection{関連研究}



\subsection{本論文の構成}
本論文の構成について説明する.
第\ref{sec:method1}章ではロボットの走行実験に必要な地形データを効率的に収集する手法について述べる.
第\ref{sec:method2}章では収集した地形を用いたロボットの走行実験の方法について述べる.
第\ref{sec:conclusion}章ではまとめと今後の課題について述べる.
\\\\
